\section{Appendix: Probability to Detect K Samples}

Let us assume we are trying to detect a specific haplogroup
within a population of N individuals. The haplogroup itself
is represented by K individuals. How big is the probability
to detect one of the K haplogroup members with S samples?

For a single sample the answer is easy:

\begin{equation}
p(1) = \frac{K}{N}
\end{equation}

And the probability, not to detect one of the K individuals
is

\begin{equation}
\tilde{p}(1) = \frac{N-K}{N}
\end{equation}

\begin{tabular}{ll}
$p(S)$         & probability to detect K with S samples\\
$\tilde{p}(S)$ & probability not to detect K with S samples\\
$N$: & number of individuals\\
$K$: & number of persons to detect\\
$S$: & number of samples
\end{tabular}
\vspace{1em}

Suppose our single sample test did not yield a positive result and
we try a second time. The previously tested individual
does not count anymore. So we must subtract one from the number
of individuals and one from the number of individuals who are 
not a member of our haplogroup. Thus the probability not to detect
one of the K individuals the second time is

\begin{equation}
\tilde{p} = \frac{N-K-1}{N-1}
\end{equation}

The maximum number of failures is $N-K$ because then only bearers
of the haplogroup we are looking for are left and the next test
will definitely a success.

To calculate the total probability of S failures we can multiply
the probabilities of failure for each single try.

\begin{eqnarray}
\tilde{p}(S) & = & \prod^{S-1}_{i=0} \tilde{p}_i\\
             & = & \prod^{S-1}_{i=0} \frac{N-K-i}{N-i}\\
             & = & \frac{(N-K)! / (N-K-S)!}{N! / (N-S)!}\\
             & = & \frac{(N-K)! (N-S)!}{N! (N-K-S)!}
\end{eqnarray}


Because the total probability must always be one, the probability
to detect K individuals with S samples is

\begin{eqnarray}
p(S) & = & 1 -\tilde{p}(S)\\
     & = & 1 - \frac{(N-K)! (N-S)!}{N! (N-K-S)!}
\end{eqnarray}













