\section{Theory}

Let us assume that we have a closed population and take a
number of genetic samples. We analyse the samples and get
different genetic lineages as a result.

The number of genetic lineages is proportional to the 
total population size.

\begin{equation}
L \sim P \label{lineages-population}
\end{equation}

\begin{tabular}{ll}
L:   & number of detected lineages\\
P:   & population size
\end{tabular}
\vspace{1em}

If we detect all male lineages, for example, the population
size would be about twice as much.

How does population size change with time? If we look after a
short period of time we may take the old population size, add
the number of births and subtract the number of deaths.

\begin{equation}
P(t+dt) = P(t) + b(t)dt - d(t)dt 
\end{equation}

\begin{tabular}{ll}
$P$: & population size\\
$b$: & number of births per second\\
$d$: & number of deaths per second\\
$t$: &time
\end{tabular}
\vspace{1em}

The probability of a mutation is the same for all newborn
babies. This means the number of newly appearing mutations
and lineages within a period of time scales with the number
of births within that period. Thus we are interested in the
number of births and rewrite the previous formula as

\begin{equation}
b(t)dt = P(t+dt) - P(t) + d(t)dt
\end{equation}

or

\begin{eqnarray}
b(t) & = & \frac{P(t+dt)- P(t)}{dt} + d(t)\\
     & = & \frac{dP(t)}{dt} + d(t) \label{births-per-second}
\end{eqnarray}

As an easy model let us assume that the number of deaths
per second is proportional to the total population size.
In this case the following equation holds:

\begin{equation}
d(t) = c_{d} P(t)
\end{equation}

\begin{tabular}{ll}
$d$:     & number of deaths per second\\
$P$:     & population size\\
$c_{d}$: & constant
\end{tabular}
\vspace{1em}

Under this assumption we can rewrite \ref{births-per-second}
as

\begin{equation}
b(t) = \frac{dP(t)}{dt} + c_{d} P(t)
\end{equation}

This means that the number of births per second depends on
two terms, the population growth and the population size.









