\section{Theory}

Let us assume that we have a closed population and take a
number of genetic samples. We analyse the samples and get
different genetic lineages as a result.

The number of genetic lineages is proportional to the 
total population size.

\begin{equation}
L \sim P \label{lineages-population}
\end{equation}

\begin{tabular}{ll}
L:   & number of detected lineages\\
P:   & population size
\end{tabular}
\vspace{1em}

If we detect all male lineages, for example, the population
size would be about twice as much.

We want to know how the number of lineages changes with time.
For this we pinpoint the number of lineages at a specific
time $t_0$ and look at infinitesimal changes.

\begin{equation}
L(t_0 + dt) = L(t_0) + dL
\end{equation}

\begin{tabular}{ll}
L:     & number of detected lineages\\
$t_0$: & arbitrary point in time\\
dt:    & infinitesimal change in time\\
dL:    & infinitesimal change in the number of lineages
\end{tabular}
\vspace{1em}

What can we say about the change in the number of lineages
$dL$? We know for sure that at every given time some
lineages disappear because they die out and others appear
because of newborn babies. Thus we may write

\begin{equation}
dL = dL_{new} - dL_{dis}
\end{equation}

\begin{tabular}{ll}
dL:     & change in the number of lineages\\
$dL_{new}$: & newly appearing lineages\\
$dL_{dis}$: & disappearing lineages\\
\end{tabular}
\vspace{1em}

This is exactly the same as

\begin{equation}
dL_{new} = dL + dL_{dis} \label{dl-new}
\end{equation}

As an easy model let us assume that the number of disappearing
lineages is proportional to the total number of lineages or
in other words: The fraction of disappearing lineages is always
the same. In this case the following equation holds:

\begin{equation}
dL_{dis} = c_{dis}L\,dt
\end{equation}

\begin{tabular}{ll}
$dL_{dis}$: & infinitesimal amount of disappearing lineages\\
$L$:       & total number of lineages\\
$dt$:      & infinitesimal amount of time\\
$c_{dis}$: & constant
\end{tabular}
\vspace{1em}

With this we can rewrite \ref{dl-new} as

\begin{equation}
dL_{new} = dL + c_{dis}L\,dt
\end{equation}

or

\begin{equation}
\frac{dL_{new}}{dt} = \frac{dL}{dt} + c_{dis}L \label{dl-new-change}
\end{equation}

\begin{tabular}{ll}
$\frac{dL_{new}}{dt}$: & change in newly appearing lineages\\
$\frac{dL}{dt}$:       & change in in the total number of lineages\\
$L$:                   & total number of lineages\\
$c_{dis}$:             & constant
\end{tabular}
\vspace{1em}

We already know that the number of lineages is proportional to
the population size \ref{lineages-population}. Thus we can
interpret the previous formula in the following way:

The change in newly appearing lineages depends on two terms:
the population growth rate and the population size.

Phylogrowth measures the change in newly appearing lineages
(actually it measures a fraction of it, but I skip the details
for now).











